% Created 2016-10-04 Tue 23:54
\documentclass[11pt]{article}
\usepackage[utf8]{inputenc}
\usepackage{lmodern}
\usepackage[T1]{fontenc}
\usepackage{fixltx2e}
\usepackage{graphicx}
\usepackage{longtable}
\usepackage{float}
\usepackage{wrapfig}
\usepackage{rotating}
\usepackage[normalem]{ulem}
\usepackage{amsmath}
\usepackage{textcomp}
\usepackage{marvosym}
\usepackage{wasysym}
\usepackage{amssymb}
\usepackage{amsmath}
\usepackage[version=3]{mhchem}
\usepackage[numbers,super,sort&compress]{natbib}
\usepackage{natmove}
\usepackage{url}
\usepackage{minted}
\usepackage{underscore}
\usepackage[linktocpage,pdfstartview=FitH,colorlinks,
linkcolor=blue,anchorcolor=blue,
citecolor=blue,filecolor=blue,menucolor=blue,urlcolor=blue]{hyperref}
\usepackage{attachfile}
\usepackage[left=1in, right=1in, top=1in, bottom=1in, nohead]{geometry}
\geometry{margin=1.0in}
\usepackage{amsmath}
\usepackage{siunitx}
\usepackage{graphicx}
\usepackage{epstopdf}
\usepackage{fancyhdr}
\usepackage{hyperref}
\usepackage[labelfont=bf]{caption}
\usepackage{setspace}
\usepackage{sectsty}
\subsectionfont{\rm}
\setlength{\headheight}{5.2pt}
\setlength{\headsep}{14pt}
\def\dbar{{\mathchar'26\mkern-12mu d}}
\pagestyle{fancy}
\fancyhf{}
\renewcommand{\headrulewidth}{0.5pt}
\renewcommand{\footrulewidth}{0.5pt}
\lfoot{\today}
\cfoot{\copyright\ 2016 W.\ F.\ Schneider}
\rfoot{\thepage}
\rhead{\bf{ND CBE 20255}}
\lhead{\bf{Exam 1}}
\chead{\bf{Spring 2016}}
\setcounter{secnumdepth}{3}
\author{William F. Schneider}
\date{\today}
\title{CBE 60553 Outline}
\begin{document}

\begin{options}
\end{options}

\
\vspace{2cm}
\begin{figure}[h]
\centering
\includegraphics[width=0.4\textwidth]{./centered-2c-NDmark.pdf}
\end{figure}
\begin{center}
{\LARGE\bf Introduction to Chemical Engineering\\(CBE 20255)}
\vspace{0.5cm}

{\Large Prof. William F.\ Schneider}
\end{center}
\vspace{2cm}
\noindent\large{{\bf NAME (PRINT):}}\_\_\_\_\_\_\_\_\_\_\_\_\_\_\_\_\_\_\_\_\_\_\_\_\_\_\_\_\_\_\_\_\_\_\_\_\_\_

\vspace{1cm}
\begin{spacing}{1.2}
\begin{tabular}{|p{5.5in}|}
\hline
{\em AS A MEMBER OF THE NOTRE DAME COMMUNITY, I WILL NOT PARTICIPATE IN OR
TOLERATE ACADEMIC DISHONESTY } \\
\hline
\end{tabular}
\end{spacing}
\vspace{1.5cm}

\noindent\large{{\bf SIGNED:}} \_\_\_\_\_\_\_\_\_\_\_\_\_\_\_\_\_\_\_\_\_\_\_\_\_\_\_\_\_\_\_\_\_\_\_\_\_\_\_\_\_\_\_\_

\vspace{1cm}
\noindent{\bf PLEASE SHOW YOUR WORK.  CLEARLY DEMONSTRATE YOUR SOLUTION PROCEDURE
AND STATE ANY ASSUMPTIONS YOU MAKE.  WRITE YOUR SOLUTIONS IN THE SPACE
PROVIDED.  BLANK PAGES ARE INCLUDED TO PROVIDE MORE ROOM FOR YOUR
WORK.  ASK THE PROCTOR IF YOU NEED ADDITIONAL SCRATCH PAPER.}
\newpage


\section{It all adds up (25 pts) \label{Q:ideal}}
\label{sec-1}
Two liquids are said to form an \emph{ideal mixture} if the volume of a mixture of the two is equal to the sum of the volumes of the individual components.  An ideal mixture has the property that:

\[ \frac{1}{\rho} = \sum_{i} \frac{x_{i}}{\rho_{i}} \]

\noindent where \(\rho\) is the density of the mixture, \(\rho_{i}\) is the density of component \emph{i}, and \(x_{i}\) is the \emph{mass fraction} of component \emph{i}.  Following is some data for mixtures of ethanol and hexane:

\begin{center}
\begin{tabular}{lcc}
\hline
 & ethanol wt \% & density (g/cm\(^{3}\))\\
\hline
Mixture 1 & 10.0 & 0.68\\
Mixture 2 & 90.0 & 0.78\\
\hline
\end{tabular}
\end{center}

\subsection{(10 pts) The gauge pressure at the bottom of a \SI{1.00}{\meter} tall drum of some ethanol/hexane mixture is \SI{7120}{\pascal}.  What is the density of the mixture?}
\label{sec-1-1}

\newpage
\noindent \textbf{Problem 1 continued}
\subsection{(15 pts) Estimate the composition (wt \%) of the mixture in the drum, assuming ethanol and hexane form an ideal mixture.}
\label{sec-1-2}

\newpage
\section{All mixed up (45 pts)}
\label{sec-2}
Two liquid streams flow into a \SI{2000.}{\liter} tank.  One stream is
10.0\%(w/w) ethanol and balance hexane, while the other is just the opposite,
10.0\%(w/w) hexane in ethanol. The densities of these two streams is given in
Problem \ref{Q:ideal}. It takes 22 minutes for the two streams combined to fill
the tank.  When it is full, the tank is found to contain a 60.0\%(w/w) ethanol
mixture with a density of \SI{0.74}{\grams\per\centimeter\cubed}.

\subsection{(10 pts) Make a sketch of this process, labeling all flows and compositions and identifying any unknowns by placing a box around them.}
\label{sec-2-1}
\vspace{10cm}
\subsection{(5 pts) Is this process steady-state or transient?}
\label{sec-2-2}
\vspace{1.5cm}
\subsection{(5 pts) What is the name of this type of process?}
\label{sec-2-3}
\vspace{1.5cm}
\subsection{(5 pts) Below is the general material balance expression.  Identify which terms are zero, which are non-zero, and the signs of the non-zero terms for the species in this process.}
\label{sec-2-4}
\vspace{0.5cm}
\begin{center}
output = input + generation - consumption - accumulation
\end{center}
\newpage
\noindent \textbf{Problem 2 continued}
\subsection{(10 pts) Write down integral mass balance expressions for ethanol and for hexane.  Your mass balance expressions should have the same two unknowns as in your diagram above.}
\label{sec-2-5}
\vspace{10cm}
\subsection{(10 pts) Solve your mass balances for the two unknowns.}
\label{sec-2-6}
\newpage
\section{You gotta keep them separated (30 pts)}
\label{sec-3}
A 50\% mixture by weight of ethanol and hexane is introduced continuously into a steady-state separator.  One outlet stream gets 60\% (by weight) of the inlet ethanol and 50\% of the inlet hexane while a second outlet contains the remainder of each.

\subsection{(10 pts) Make a sketch of this process, labeling all flows and compositions and identifying unknowns by drawing boxes around them. Assume an inlet basis of 100 kg/s.}
\label{sec-3-1}
\vspace{10cm}
\subsection{(5 pts) Count the number of unknowns and identify the equations and other pieces of information that relate them.  How many degrees of freedom does the system have?}
\label{sec-3-2}
\newpage
\subsection{(15 pts) Based on your degree of freedom analysis, determine all the flow rates and compositions that can be determined.}
\label{sec-3-3}
% Emacs 25.0.50.1 (Org mode 8.2.10)
\end{document}